\section{Aspects of user visualization}

\subsection{Amount of contribution}
\subsubsection{Theory}
\subsubsection{Visualization}

\subsection{Topic Models}
\subsubsection{Latent dirichlet allocation}

Latent Dirichlet allocation (LDA) \cite{Blei:2003:LDA:944919.944937, heinrich2005parameter, steyvers} is a unsupervised machine learning technique 
that discovers latent topics from a corpus of documents so that the documents 
can then be assigned automatically into appropriate topics. New documents can 
also be classified into topics based on these latent topics. 
More specifically, in LDA, each document is represented as a mixture of various
topics, where each topic is a mixture of words. These mixtures are represented
by P(topic|document) for all topics and documents and P(word|topic) for all words 
in the vocabulary. Each word may occur in several different topics with a different probability, and each document is assumed to be characterized by a particular set of topics.


\subsubsection{Nonnegative Matrix Factorization}

Nonnegative matrix factorization (NMF) is an unsupervised family of algorithms that simultaneously perform dimension reduction and clustering.NMF was first introduced
by Paatero and Tapper \cite{paa94} as positive matrix factorization and subsequently
popularized by Lee and Seung \cite{lee99}.


\subsubsection{Compare LDA and NFM}

\subsubsection{Visualization}


\subsection{Impact of post}
\subsubsection{Theory}

\subsubsection{Cosine Similarity}

For creating the user timeline, we need to find similar posts that a user share both in Twitter and Facebook. For this purpose, we calculate the cosine similarity metric.The cosine similarity \cite{similarity} between two vectors (or two documents on the Vector Space) is a measure that calculates the cosine of the angle between them. This metric is a measurement of orientation and not magnitude, it can be seen as a comparison between documents on a normalized space because we’re not taking into the consideration only the magnitude of each word count (tf-idf) of each document, but the angle between the documents. 

Given two posts $t_{a}$ and $t_{b}$, their cosine similarity is

\begin{equation}
\cos ({\bf t_{a}},{\bf t_{b}})= {{\bf t_{a}} {\bf t_{b}} \over \|{\bf t_{a}}\| \|{\bf \textbf{b}}\|} 
\end{equation}

where  $t_{a}$ and $t_{b}$ are $m$-dimensional vectors over the term set
$T = {t_{1}, . . . , t_{m}}$. Each dimension represents a term with its
weight in the document, which is non-negative. As a result, the cosine similarity 
is non-negative and bounded between $[0,1]$. 

We predefine a threshold to accept two similar posts to have similarity at least $60\%$. This portion lets us a great amount of similar posts and also recognize twitter posts that have urls, hashtags and mentions.



\subsubsection{Visualization}
\subsubsection{Other possibilities to visualize text corpus}
%%% Local Variables: 
%%% mode: latex
%%% TeX-master: "isae-report-template"
%%% End: 